\section{Liste}

\begin{frame}[fragile]
\frametitle{Liste}
    \begin{block}{Liste}
Le liste sono delle variabili che, invece che contenere dei singoli valori ne contengono molteplici.
    \end{block}
    
    \begin{lstlisting}
teachers =  ['Chiara', 'Enrico', 'Sofia']
answers = ['Si', 'No']
vocals = ['a', 'e', 'i', 'o', 'u']
    \end{lstlisting}
\end{frame}

\begin{frame}[fragile]
\frametitle{Liste}
    \begin{block}{Come leggiamo i valori di una lista}
Pensiamo ad una lista come ad un mobile con dei cassetti numerati da 0 (primo cassetto) a lungezza della lista -1 (ultimo cassetto).

! Attenzione, stiamo partendo da 0 !
    \end{block}
    
    \begin{lstlisting}
vocals = ['a', 'e', 'i', 'o', 'u']
    \end{lstlisting}
    
        \begin{block}{}
Per aprire i "cassetti" e leggere i singoli valori della lista quindi faremo
    \end{block}

    \begin{lstlisting}
vocals[0] # Avra' valore 'a'
vocals[1] # Avra' valore 'e'
vocals[2] # ...
vocals[3]
vocals[4]
    \end{lstlisting}

\end{frame}

\begin{frame}[fragile]
\frametitle{Liste}
    \begin{block}{Come scriviamo i valori di una lista}
Similmente a come li leggiamo e a come assegnamo un valore ad una variabile normale.
    \end{block}
    
    \begin{lstlisting}
answers = ['Si', 'No']

answers[0] = 'Yes' # Nella posizione 0 della lista scrivo 'Yes', sostituendolo al 'Si'
    \end{lstlisting}

    \begin{block}{append()}
La funzione append() aggiunge valori ad una lista gia' definita
    \end{block}

    \begin{lstlisting}
answers = ['Si', 'No']

answers.append('Forse') # Ora answers avra' valori ['Si', 'No', 'Forse']
    \end{lstlisting}

\end{frame}

\begin{frame}[fragile]
\frametitle{Liste}
    \begin{block}{Lista vuota}
Adesso che conosciamo append() possiamo anche crearci delle lista vuote e all'occorrenza aggiungere valori
    \end{block}
    
    \begin{lstlisting}
answers = []
answers.append('Si')
answers.append('No')
    \end{lstlisting}

    \begin{block}{reverse()}
La funzione reverse() inverte l'ordine degli elementi di una lista, modificando la lita stessa!

! Si, la lista e' ordinata !
    \end{block}

    \begin{lstlisting}
vocals = ['a', 'e', 'i', 'o', 'u']
vocals.reverse() # ['u', 'o', 'i', 'e', 'a']
    \end{lstlisting}

\end{frame}

\begin{frame}[fragile]
\frametitle{Liste}
    \begin{block}{split()}
Un altra funzione utile e' split().
Data una frase, un insieme di parole separate da spazi, split() restituisce una lista fatta dalle parole della frase stessa
    \end{block}
    
    \begin{lstlisting}
sentence = 'Oggi a Cesena nevica e fa un gran caldo'
sentence.split()
# ['Oggi', 'a', 'Cesena', 'nevica', 'e', 'fa', 'un', 'gran', 'caldo']
splittedSentence = sentence.split()
splittedSentence[3] # Quale valore avra'?
    \end{lstlisting}

\end{frame}

\begin{frame}[fragile]
\frametitle{Liste}
    \begin{block}{Cosa succede se...}
provo a leggere in una posizione maggiore della lunghezza della lista?
    \end{block}
    
    \begin{lstlisting}
vocals = ['a', 'e', 'i', 'o', 'u']
print(vocals[999])
print(vocals[10])
    \end{lstlisting}

\end{frame}

\begin{frame}[fragile]
\frametitle{Liste}
    \begin{block}{Cosa succede se...}
provo a leggere o scrivere in una posizione maggiore della lunghezza della lista?
    \end{block}
    
    \begin{lstlisting}
vocals = ['a', 'e', 'i', 'o', 'u']
print(vocals[999])
'vocals[9999]
IndexError: list index out of range'

vocals[10] = 'abcdefghi...'
'IndexError: list assignment index out of range'
    \end{lstlisting}
\end{frame}

\begin{frame}[fragile]
\frametitle{Liste}
    \begin{block}{range()}
    La funzione range restituisce una lista di numeri interi compresi nell'intervallo specificato come parametro
    \end{block}
    
    \begin{lstlisting}
range(5) # [0, 1, 2, 3, 4]
# Se non specifico il minimo parto da 0    
range(0, 10) # [0, 1, 2, 3, 4, 5, 6, 7, 8, 9]
range(5, 10) # [5, 6, 7, 8, 9] ! Il 10 non c'e'
    \end{lstlisting}
\end{frame}

\begin{frame}[fragile]
\frametitle{Liste}
    \begin{block}{list()}
    La funzione list prende come parametro un valore e restituisce una lista fatta degli elementi che compongono il parametro
    \end{block}
    
    \begin{lstlisting}
list('Ciao') # ['C', 'i', 'a', 'o']
    \end{lstlisting}
\end{frame}

\begin{frame}[fragile]
\frametitle{Liste}
    \begin{block}{Operatore in}
    L'operatore \textbf{in} ci dice se un determinato elemento sia contenuto in una lista o meno.
    \end{block}
    
    \begin{lstlisting}
    vocals = ['a', 'e', 'i', 'o', 'u']
    'e' in vocals # True
    'g' in vocals # False
    \end{lstlisting}
\end{frame}
